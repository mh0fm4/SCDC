\section{Access: File system storage}
\label{sect:dataprov_access}

\begin{itemize}
  \item file system storage
  \item backends: local file system, NFS, WebDAV
\end{itemize}


\subsection{Server-side setup}

\begin{description}
  \item[\texttt{access:fs}] Parameter: root directory
  \item[\texttt{access:nfs}] Parameter: NFS export URL \texttt{nfs://<host>/<path>}
  \item[\texttt{access:webdav[:username:password]}] Parameter: WebDAV server URL \texttt{http://<host>[:port]/<path>}\\
if the optional config switches \text{username} and \text{password} are given, then the user name and the password have to given as separate string parameters
\end{description}


\subsection{Client-side usage}

\begin{dscmdref}

% command: cd
\refcmd{cd [|<dir>]}{
\refcmduse{
Change to directory given by \dscmd{<dir>}.
}
% \refcmdparam{[|<dir>]}
\refcmdfail{Store \dscmd{<dir>} does not exist.}
}

% command: ls
\refcmd{ls}{
\refcmduse{List entries of the current directory.}
\refcmdres{Number and list of entries separated by \enquote{|} character.}
}

% command: info
\refcmd{info [|<entry>]}{
\refcmduse{General information about the current directory or the given directory/file <entry>.}
% \refcmdparam{[|<entry>]}
\refcmdres{Directory or file information.}
}

% command: mkd
\refcmd{mkd <dir>}{
\refcmduse{Create directory <dir> within the current directory.}
% \refcmdparam{<dir>}
\refcmdfail{Directory \dscmd{<dir>} could not be created.}
}

% command: put
\refcmd{put <file> [<pos>][:<size>]}{
\refcmduse{
Write data from the command input to the file \dscmd{<file>} of the current directory.
Data is written at the beginning of the file or (optionally) at position \dscmd{<pos>}.
Either all input or (optionally) at most \dscmd{<size>} bytes are written.
If the entry already exists, its data is overwritten.
If the entry does not exists, it is created.
}
% \refcmdparam{<file> [<pos>][:<size>]}
\refcmdin{Data to be written to the file.}
\refcmdfail{File \dscmd{<file>} could not be created or data write failed.}
}

% command: get
\refcmd{get <file> [<pos>][:<size>]}{
\refcmduse{
Read data from the file \dscmd{<file>} of the current directory into the command output.
Data is read at the beginning of the file or (optionally) at position \dscmd{<pos>}.
Either all data or (optionally) at most \dscmd{<size>} bytes are read.
If the file does not exists, then the command fails.
}
% \refcmdparam{<file> [<pos>][:<size>}
\refcmdout{Data read from the file.}
\refcmdfail{Entry \dscmd{<entry>} does not exist or data read failed.}
}

% command: rm
\refcmd{rm [<entry>]}{
\refcmduse{Remove the directory/file <entry> of the current directory.}
% \refcmdparam{<entry>}
\refcmdfail{Directory/file \dscmd{<entry>} does not exist or could not be removed.}
}

\end{dscmdref}


\subsection{Configuration}

\begin{description}
  \item[\texttt{access:fs}] none
\end{description}
