\section{Store: Non-hierarchical folder-oriented storage}
\label{sect:dataprov_store}

\begin{itemize}
  \item non-hierarchical folder-oriented storage
  \item backends: memory, local file system, MySQL, NFS, WebDAV
\end{itemize}


\subsection{Server-side setup}

\begin{description}
  \item[\texttt{store:mem}] Parameter: none
  \item[\texttt{store:fs}] Parameter: root directory
  \item[\texttt{store:nfs}] Parameter: NFS export URL \texttt{nfs://<host>/<path>}
  \item[\texttt{store:webdav[:username:password]}] Parameter: WebDAV server URL \texttt{http://<host>[:port]/<path>}\\
if the optional config switches \text{username} and \text{password} are given, then the user name and the password have to given as separate string parameters
\end{description}


\subsection{Client-side usage}

\begin{dscmdref}
% command: cd
\refcmd{cd [|ADMIN|<store>]}{
\refcmduse{
Change to administration mode using no parameter or parameter \dscmd{ADMIN}.
Select a single store given by \dscmd{<store>}.
}
% \refcmdparam{[|ADMIN|<store>]}
\refcmdfail{Store \dscmd{<store>} does not exist.}
}
\end{dscmdref}


\subsubsection*{Administration mode}

\begin{dscmdref}

% command: info
\refcmd{info}{
\refcmduse{General information about the stores.}
\refcmdres{Store information.}
}

% command: ls
\refcmd{ls}{
\refcmduse{List all available stores.}
\refcmdres{Number of stores and list of store names separated by \enquote{|} character.}
}

% command: put
\refcmd{put <store>}{
\refcmduse{Create a store \dscmd{<store>}.}
% \refcmdparam{<store>}
\refcmdfail{Store \dscmd{<store>} could not be created.}
}

% command: rm
\refcmd{rm <store>}{
\refcmduse{Remove the store \dscmd{<store>}.}
% \refcmdparam{<store>}
\refcmdfail{Store \dscmd{<store>} does not exist or could not be removed.}
}

\end{dscmdref}

\subsubsection*{Single store selected}

\begin{dscmdref}

\refcmd{info}{
\refcmduse{General information about the selected store.}
\refcmdres{Store information.}
}

\refcmd{ls}{
\refcmduse{List all available entries of the selected store.}
\refcmdres{Number of entries and list of entry names separated by \enquote{|} character.}
}

% command: put
\refcmd{put <entry> [<pos>][:<size>]}{
\refcmduse{
Write data from the command input to the entry \dscmd{<entry>} of the selected store.
Data is written at the beginning of the entry or (optionally) at position \dscmd{<pos>}.
Either all input or (optionally) at most \dscmd{<size>} bytes are written.
If the entry already exists, its data is overwritten.
If the entry does not exists, it is created.
}
% \refcmdparam{<entry> [<pos>][:<size>]}
\refcmdin{Data to be written to the entry.}
\refcmdfail{Entry \dscmd{<entry>} could not be created or data write failed.}
}

% command: get
\refcmd{get <entry> [<pos>][:<size>]}{
\refcmduse{
Read data from the entry \dscmd{<entry>} of the selected store into the command output.
Data is read at the beginning of the entry or (optionally) at position \dscmd{<pos>}.
Either all data or (optionally) at most \dscmd{<size>} bytes are read.
If the entry does not exists, then the command fails.
}
% \refcmdparam{<entry> [<pos>][:<size>]}
\refcmdout{Data read from the entry.}
\refcmdfail{Entry \dscmd{<entry>} does not exist or data read failed.}
}

% command: rm
\refcmd{rm <store>}{
\refcmduse{Remove the entry \dscmd{<entry>} of the selected store.}
% \refcmdparam{<store>}
\refcmdfail{Entry \dscmd{<entry>} does not exist or could not be removed.}
}

\end{dscmdref}


\subsection{Configuration}

\begin{description}
  \item[\texttt{store:mem}] none
  \item[\texttt{store:fs}] none
\end{description}
