\section{SCDC overview}

A software component can be represented by any program or function available for the execution of a specific task.
The library implements a service-oriented usage model where active client components access passive service components.
Within this model, a software component might act in several roles, for example, by providing multiple different services or by being both a service and a client.
The library design is application-independent and any interaction between a client and a services is considered as a data transfer.

To act as a service with a specific functionality, a software component provides so-called \textit{datasets} which are managed by \textit{data providers}.
Data providers and datasets are identified with an URI-based addressing scheme as follows:
\begin{alltt}
  <access-method>://<component-id>/<data-provider>[/<dataset>]
\end{alltt}
\begin{description}

  \item[\texttt{<access-method>}] specifies how the service component, which provides the datasets, should be accessed and how data to and from the component should be transferred.
Examples are \texttt{scdc} for accessing a local service component within the same process directly with function calls or \texttt{scdc+tcp} for accessing a remote service component in a distributed environment with TCP/IP network communication.
Direct access is enabled by default for any component.
Further access methods can be explicitly enabled by setting up so-called \textit{node ports} using the corresponding library functions as described in Sect.~\ref{sect:server_nodeport}.

  \item[\texttt{<component-id}] identifies the service component to be accessed.
The specific format of the identifier depends on the access method to be used.
For example, direct access the local service component (\ie, method \texttt{scdc}) uses an empty identifier while accessing a remote service component with TCP/IP network communication (\ie, method \texttt{scdc+tcp}) uses the hostname (or IP address) of the computer where the remote service component is executed as identifier.
Further details about specific access methods and their functionalities are described in Sect.~\ref{chap:nodeport}.

  \item[\texttt{<data-provider>}] identifies the data provider to be accessed on a service component.
The identifier is an alphanumeric string (\ie, without further path hierarchy).
A service component can contain several data providers distinguished by different identifiers.
Data providers can be created by using the corresponding library functions as described in Sect.~\ref{sect:server_dataprov}.

  \item[\texttt{<dataset>}] identifies the dataset of a data provider to be accessed on a service component.
The specific format of the identifier depends on the data provider of the dataset.
Further details about specific data providers and their functionalities are described in Chap.~\ref{chap:dataprov}.

\end{description}

To act as a client, a software component accesses the datasets of services by executing \textit{commands}.
Each execution of a commands can involve input data transferred from the client to the service and output data transferred back from the service to the client.
Input and output data can be processed in continuous parts as a data stream, \ie in both directions the data does not need to be processed at once in a single piece, but can be continuously provided or consumed.
The corresponding library functions to be used by a client component are described in Chap.~\ref{chap:client}.

Additional information can be found in the following publications:
\begin{itemize}
  \item \bibolay[inline,plain]{HOSS14}
  \item \bibolay[inline,plain]{HR15}
\end{itemize}
