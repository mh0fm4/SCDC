\section{Functions}

Before using (almost) any library function (\eg, at the beginning of a program), the SCDC library has to be initialized using the function \fnct{scdc_init}.
After the last usage of library functions (\eg, before the ending of a program), the SCDC library has to be released using the function \fnct{scdc_release}.
The library should not be initialized or released multiple times in a row.
(However, initializing the library again after it has been released should be possible.)

\reffnc{scdcint_t scdc_init(const char *conf, ...);}{%
\reffncparam{conf}{String of colon-separated configuration parameters. The constant \cnst{SCDC_INIT_DEFAULT} or NULL can be used as default.}
\reffncparam{...}{Varying number of additional parameters (see description of supported configuration parameters below).}
\reffncret{Whether the function was successful (\cnst{SCDC_SUCCESS}) or not (\cnst{SCDC_FAILURE}).}
\reffncdesc{Function for initializing the SCDC library.
Must be called before (almost) all other library function call.
Currently, the following configuration parameters are supported:
\begin{description}
  \item[\code{no_direct}] Disable the default direct access.
\end{description}
}
}

\reffnc{void scdc_release(void);}{%
\reffncdesc{Function for releasing the SCDC library.
Must be called after (almost) all other library function calls.}
}

Library functions usually return a value signaling their success or failure.
Further (textual) information about the result of a function (\eg, a requested information or an error message) are stored internally in a static string variable.
The variable can be accessed with the function \fnct{scdc_last_result} and contains the result of the last library function that was executed.

\reffnc{const char *scdc_last_result();}{%
\reffncret{Pointer to a static C string containing the last results message.}
\reffncdesc{Function for returning the result string of the last SCDC library function call.}
}

The library can produce various logging information during its execution (if enabled at compile time, see configuration file \texttt{Makefile.in} of the build system described in Sect.~\ref{sect:intro_usage}).
The default output is \code{stdout} for tracing information and \code{stderr} for error information.
The output can be changed with the function \fnct{scdc_log_init} and reverted back to the default with the function \fnct{scdc_log_release}.
These changes also can be made before the initialization and after the release of the library (\ie, with \fnct{scdc_init} and \fnct{scdc_release}).

\reffnc{scdcint_t scdc_log_init(const char *conf, ...);}{%
\reffncparam{conf}{String of colon-separated configuration parameters.}
\reffncparam{...}{Varying number of additional parameters (see description of supported configuration parameters below).}
\reffncret{Whether the function was successful (\cnst{SCDC_SUCCESS}) or not (\cnst{SCDC_FAILURE}).}
\reffncdesc{Function for initializing the logging.
Can be called before \fnct{scdc_init}.
Currently, the following configuration parameters are supported:
\begin{description}
  \item[\code{log_handler}]
Two user-defined functions are used to output tracing and error information.
For each user-defined function, a pointer to the function and a user-defined pointer (\eg, to some user data or NULL) have to be specified as additional parameters.
The logging functions have to comply to the format defined by type \dtyp{scdc_log_handler_f} and the corresponding pointer to the user data is provided to each function call (see description below).
  \item[\code{log_FILE}]
Two \code{FILE} streams are used to output tracing and error information.
Pointers to these \code{FILE} streams have to be specified as additional parameters.
  \item[\code{log_filepath}]
Two files are used to store tracing and error information.
A string \code{<logfilepath>} containing the path and basic file name as to be specified as additional parameter.
Tracing information are stored in file \code{<logfilepath>.out} and error information in file \code{<logfilepath>.err}.
\end{description}
}
}

\reffnc{void scdc_log_release(void);}{%
\reffncdesc{Function for releasing the logging.
Can be called after \fnct{scdc_release}.}
}

User-defined functions for the output of tracing and error information have to comply to the following format:\\
\textit{Format:} \reffnc{scdcint_t scdc_log_handler_f(void *data, const char *buf, scdcint_t buf_size);}{%
\reffncparam{data}{User-defined pointer previously specified together with the corresponding user-defined logging function.}
\reffncparam{buf}{C string containing the logging information.}
\reffncparam{buf_size}{Length of the logging information given in \prmt{buf} (without terminating null character).}
\reffncret{Number of characters written to the output.}
}
