\section{Node port management}
\label{sect:server_nodeport}

Node ports represent the access methods through which a service component can be accessed by a client component.
Direct access to a local service component within the same process with function calls is enabled by default.
Further access methods can be enabled by opening new node ports with the function \fnct{scdc_nodeport_open}.
An open node port is identified by a handle of type \dtyp{scdc_nodeport_t}, which can be used for
starting and stopping accepting connections with the functions \fnct{scdc_nodeport_start} and \fnct{scdc_nodeport_stop},
canceling a running node port (\ie, when started in a blocking mode) with the function \fnct{scdc_nodeport_cancel},
and to close a node port with the functions \fnct{scdc_nodeport_close}.
A node port can be started and stopped multiple times as long as it is not closed.
Furthermore, node ports should always be properly stopped (if started) and closed to allow for an appropriate release of utilized system resources.
Details about specific access methods available are described in Chapter~\ref{chap:nodeport}.

\reffnc{scdc_nodeport_t scdc_nodeport_open(const char *conf, ...);}{
\reffncparam{conf}{String of colon-separated configuration parameters.}
\reffncparam{...}{Varying number of additional parameters.}
\reffncret{Handle of the created node port or \code{SCDC_NODEPORT_NULL} if the function failed.}
\reffncdesc{Function to open a new node port.
The specific access method to be used has to be specified as the first configuration parameter given in parameter \prmt{conf}.
Currently, the following access methods are supported:
\reflst{
\reflstcode{direct} Direct function calls (see Sect.~\ref{sect:nodeport_direct}).
\reflstcode{uds} Unix domain socket (see Sect.~\ref{sect:nodeport_uds}).
\reflstcode{tcp} TCP/IP network communication  (see Sect.~\ref{sect:nodeport_tcp}).
\reflstcode{mpi} Message Passing Interface (see Sect.~\ref{sect:nodeport_mpi}).
}
}
}

\reffnc{void scdc_nodeport_close(scdc_nodeport_t nodeport);}{
\reffncparam{nodeport}{Handle of the node port.}
\reffncdesc{Function to close a node port.}
}

\reffnc{scdcint_t scdc_nodeport_start(scdc_nodeport_t nodeport, scdcint_t mode);}{
\reffncparam{nodeport}{Handle of the node port.}
\reffncparam{mode}{Specification of the mode for starting the node port (see description below).}
\reffncret{Whether the function was successful (\cnst{SCDC_SUCCESS}) or not (\cnst{SCDC_FAILURE}).}
\reffncdesc{Function to start accepting connections by a node port.
The parameter \prmt{mode} has to be used to specify the mode for running the node port.
Currently, the following modes specified with predefined constants are supported:
\reflst{
\reflstcode{SCDC_NODEPORT_START_NONE}{Do not run the node port.}
\reflstcode{SCDC_NODEPORT_START_LOOP_UNTIL_CANCEL}{Run the node port in a blocking way (\ie, function blocks) until canceled with function \fnct{scdc_nodeport_cancel}.}
%\reflstcode{SCDC_NODEPORT_START_LOOP_UNTIL_IDLE}{}
\reflstcode{SCDC_NODEPORT_START_ASYNC_UNTIL_CANCEL}{Run the node port in a non-blocking way (\ie, function returns immediately) until canceled with function \fnct{scdc_nodeport_cancel}.}
%\reflstcode{SCDC_NODEPORT_START_ASYNC_UNTIL_IDLE}{}
}
}
}

\reffnc{scdcint_t scdc_nodeport_stop(scdc_nodeport_t nodeport);}{
\reffncparam{nodeport}{Handle of the node port.}
\reffncret{Whether the function was successful (\cnst{SCDC_SUCCESS}) or not (\cnst{SCDC_FAILURE}).}
\reffncdesc{Function to stop accepting connections by a node port.}
}

\reffnc{scdcint_t scdc_nodeport_cancel(scdc_nodeport_t nodeport, scdcint_t interrupt);}{
\reffncparam{nodeport}{Handle of the node port.}
\reffncparam{interrupt}{Specifies whether the cancellation should be performed \enquote{hard} (1) or \enquote{soft} (0).}
\reffncret{Whether the function was successful (\cnst{SCDC_SUCCESS}) or not (\cnst{SCDC_FAILURE}).}
\reffncdesc{Function to cancel accepting connections by a node port (\ie, if the node port was started in in a blocking mode).}
}

The following auxiliary functions are provided to ease the handling of node port-specific parts of URI addresses.
The function \fnct{scdc_nodeport_authority} can be used to construct the authority part of an URI address for a specific access method.
The function \fnct{scdc_nodeport_supported} can be used to test whether the library supports the specific access method given by an URI address.

\reffnc{const char *scdc_nodeport_authority(const char *conf, ...);}{
\reffncparam{conf}{String of colon-separated configuration parameters.}
\reffncparam{...}{Varying number of additional parameters.}
\reffncret{Pointer to a static string containing the authority information or NULL if the function failed.}
\reffncdesc{Function to construct the authority information of an URI address for a given node port and its properties.}
}

\reffnc{scdcint_t scdc_nodeport_supported(const char *uri, ...);}{
\reffncparam{uri}{String containing the pattern of the URI address. The string can include placeholders \enquote{\code{\%s}} that are replaced with strings given as additional parameters.}
\reffncparam{...}{Varying number of additional parameters.}
\reffncret{Whether the node port is supported (\cnst{SCDC_SUCCESS}) or not (\cnst{SCDC_FAILURE}).}
\reffncdesc{Function to determine whether a node port specified by the scheme of a given URI address is supported.}
}
